% This is samplepaper.tex, a sample chapter demonstrating the
% LLNCS macro package for Springer Computer Science proceedings;
% Version 2.20 of 2017/10/04
%
\documentclass[runningheads]{llncs}
%
\usepackage{graphicx}
\usepackage{float}
% Used for displaying a sample figure. If possible, figure files should
% be included in EPS format.
%
% If you use the hyperref package, please uncomment the following line
% to display URLs in blue roman font according to Springer's eBook style:
% \renewcommand\UrlFont{\color{blue}\rmfamily}

\begin{document}
%
\title{Automatic Concept Explanation for Deaf Users}
%
%\titlerunning{Abbreviated paper title}
% If the paper title is too long for the running head, you can set
% an abbreviated paper title here
%
\author{André Dias\inst{1} \and
Nuno Escudeiro\inst{2}}
%
\authorrunning{A. Dias, N. Escudeiro}
% First names are abbreviated in the running head.
% If there are more than two authors, 'et al.' is used.
%
\institute{Instituto Superior de Engenharia do Porto\\
\email{1140279@isep.ipp.pt}\and
Instituto Superior de Engenharia do Porto\\
\email{nfe@isep.ipp.pt}}
%
\maketitle              % typeset the header of the contribution
%
\begin{abstract}
    Concept explanation can take a fundamental part of the learning process.
    Its importance is even greater when it comes to languages with small lexicons, such as sign languages.
    In Portugal the official language used by the deaf community and the community that surrounds them is known as Língua Gestual Portuguesa (LGP).
    This language like other sign languages has a small lexicon that is not able to represent all the word of the Portuguese language.
    The users of LGP when faced with words like 'Nanotechnology' have to resort to tools made for Portuguese because the ones available to them are not ideal.
    Those solution either lack a translation which is the case of online dictionaries or are not practical which is the case of sign language interpreters.
    With the goal to solve this problem, this solution was created with the hypothesis of being able to utilize machine learning techniques, summarization and automatic translation to generate the explanation of a given word or expression in Língua Gestual Portuguesa (LGP), that does not exist in its lexicon.
    The solution consists in using the mentioned techniques to develop an Application Programming Interface (API) capable of generating an explanation of a given word or expression.
    This API will feed a web application responsible for receiving the user input and presenting the generated explanation in plain text and its translation to LGP.
    A previously developed avatar that is integrated with the web application is responsible for said translation.
    %TODO
    (Resultados)

    \keywords{Sign language \and Text summarization \and Automatic concept explanation.}
\end{abstract}
%
%
%
\section{Introduction}

Concepts are abstract or generic ideas that have a common set of features that are shared across multiple situations and contexts.
Concept explanation can take a fundamental part of the learning process, whether it takes place in an educational environment, a day-to-day or a professional context.
The importance of concept explanation is even greater when it comes to languages with a small lexicon, in particular sign languages.

According to the World Health Organization \cite{who_2020}, as of march 2018, there were 466 million people with disabling hearing loss, this represents 5\% of the world population.
Without being able to hear, deaf people are not able to communicate using sound therefor they must use sign languages.

In Portugal the official language of the deaf community is known as Língua Gestual Portuguesa (LGP).
This language is also used by the surrounding community, such as their relatives, teaches, and so on.
The lexicon of LGP is composed by multiple signs where each one represents a word or an expression in Portuguese.
However the opposite is not true.
There are numerous words in Portuguese that don't have a sign that translates them.

In order for a LGP user to understand those words/expressions the explanation must be formed by the available lexicon, yet there are no LGP encyclopedias.
The closest thing available are the LGP dictionaries but they fail to translate words that have no sign that can directly translate them.

This problem is recurrent when it comes to scientific domains.
To understand a concept like 'Nanotechnology', which there are no signs for it, a LGP user has to access tools with content made for the Portuguese users.

Even though LGP is the language taught to the people that are not able to communicate using Portuguese, they are two very different languages.
Each with their own lexicon and grammar, so most of the LGP users will struggle to understand Portuguese content.

A LGP user might also resort to a sign language interpreter that would explain the meaning of the word to him, but this would be a even less practical solution

With the goal of solving this problem for the users of LGP this project was created with the hypothesis that is possible to utilize machine learning techniques, summarization and automatic translation to generate the explanation of a given word or expression in LGP that does not exist in its lexicon.

In order to test this hypothesis, the developed project consists of three major components.
An API capable of generating explanations of a given word.
A formula that calculates the LGP readability score of each expression.
A web app for the users to interact with, that allows them to search for a word and displays the generated explanation in plain text as well as a translation in LGP with the help of a previously developed avatar.

In regards to results it is expected that this project is able to generate accurate LGP explanation that were classified with a reliable LGP readability score by the developed formula.

After this introduction the paper is divided into four more sections:
Section 2 presents a deeper explanation of the partial solutions available as well as the Portuguese readability metrics that were the foundation of the LGP readability metric.
Section 3 describes the developed solution.
Section 4 explains this project contributions and how is it  going to be evaluated to confirm or deny the previously mentioned hypothesis.
Section 5 draws the conclusions and how it can evolve to become more complete and target a bigger audience.

\section{Related Work}

As mentioned in the introduction there is a lack for a solution capable to translate concepts from Portuguese to LGP.
There are however some partial solutions.

This section will describe those partial solutions as well as the existing Portuguese readability metrics that were analyzed in order to create the LGP readability metric.

\subsection{Online Dictionaries}

One of the most known online dictionaries with LGP content is the SpreadTheSign \cite{sts_2020}.
In this dictionary a user can search for a word and if there is a previously recorded video translation for that word, in the site database, it will be displayed for the user.
The search results only display the video of a person performing the corresponding sign, and the possibility to look at the same word in another sign language.

Another online dictionary that provides content for the LGP users is the Infopédia \cite{infopedia_2020}.
Although this is a Portuguese dictionary, it also contains a section for searching words in LGP.
Here the search results, not only provide a video translation of the word, but also an explanation in Portuguese in how to reproduce the sign shown in the video.

Online dictionaries as a solution is limited by the database of prerecorded videos, and the LGP lexicon.
The latter is a problem because they focus on a direct translation, word to sign, instead of trying to translate the meaning of the word.

\subsection{Sign Language Interpreter}

In regard to the sign language interpreters in Portugal there is CTILG \cite{ctilg_2020}, a company that provides professional LGP translation services in workshops, classes, congresses, events and more.
This company is responsible for the live translation of some morning TV shows.

A more affordable solution for a regular LGP user is the Serviin \cite{serviin_2020} which is a service that provides a interpreter to work as a middle-man between a deaf person and a targeted service/company.
This solution is available as a mobile app, with a very low cost for the deaf user, or through a  web app that is free.

This solution is limited by the interpreters own knowledge and their cost.
Also by utilizing this solution the LGP user is sacrificing some of his autonomy.

\subsection{Portuguese Readability Metrics}

Readability metrics are used to calculate a score that relates to the level of education a reader will need to fully understand the context of a given text.
There are some widely known and used metrics for the English language, such as Flesh Reading Ease, New Dale-Chall, SMOG, Flesh-Kincaid, Gunning Fog and so on.

In 2019, Antunes, H and Lopes, C published an article \cite{ptread_2019} that adapted the values of the metrics used to calculate the readability of text in English so it could be applied to Portuguese.
The adapted readability metrics are presented in table \ref{table:ptformulas}.

\begin{table}
    \caption{Adjusted Portuguese formulas.}
    \label{table:ptformulas}
    \begin{tabular}{l|l}
        \hline
        {} & {\bfseries Formula} \\
        \hline
        SMOG & \(16.830 \times \sqrt{CW \times 30 \div SE} - 23.809\)  \\
        \hline
        Flesch-Kincaid & \(0.883 \times WO \div SE + 17.347 \times SY \div WO - 41.239\) \\
        \hline
        ARI & \(6.286 \times CH \div WO + 0.927 \times WO \div SE - 36.551\) \\
        \hline
        Coleman Liau & \(5.730 \times CH \div WO - 171.365 \times SE \div WO - 6.662\) \\
        \hline
        Gunning Fog & \(0.760 \times WO \div SE + 58.600 \times CW \div WO - 12.166\) \\
        \hline
        \multicolumn{2}{l}{CH - characters, CW - complex words, SY - syllables, WO - words, SE - sentences}
    \end{tabular}
\end{table}

\section{Automatic Concept Explanation}

%TODO

In order to solve the previously mentioned problems Automatic Concept Explanation was created.
The main functionality of this solution is to allow the LGP users to search for a word or expression in Portuguese and display its explanation as well as the respective LGP translation.

A user interacts with the web application that was developed in React, an open-source JavaScript framework for building user interfaces.
Here the user can search of the desired concept as well as changing the page language.
This is also the place where the generated explanation are shown in plain text or in LGP by the avatar.

When a concept is searched for, the input is sent to a API that was developed in Flask, a lightweight open-source Python web framework.
This API will generate a list of explanations using Text Mining, Information Scraping and Information Retrieval techniques.

\dots

All the components can be seen in the figure \ref{fig1}.

\begin{figure}[H]
\centering
\includegraphics[scale=0.4]{component_diagram.png}
\caption{Component Diagram.} \label{fig1}
\end{figure}

%The figure \ref{fig2} shows the user interface when a user searches for the word 'Manga'.
%
%\begin{figure}[H]
%\centering
%\includegraphics[scale=0.2]{interface.png}
%\caption{Application Interface.} \label{fig2}
%\end{figure}


\subsection{LGP Readability}

The developed API always generates more than one explanations and the number is even higher for words that have different meanings based on its context.
This explanations need to be sorted in order to present the user with the ones best suited for their needs.

For a regular language, there are metrics that can be used to classify each expression in order to sort them accordingly.
One metric that can be used for this is the readability score.
This score indicates how easily a reader would understand the explanation that he's reading.

There are multiple formulas that could be used for calculating this score in Portuguese.
However there is none for the LGP.

After analyzing the existing formulas for calculating the readability for the PL, a new formula was developed to calculate the readability of LGP content.
The formula aims to calculate the readability of each sign using the following variables:

\begin{itemize}
    \item Number of hand configuration.
    \item Number of moments
    \item Number of hands used
    \item Number of facial expression used
\end{itemize}

The resulting formula is (INSERT FORMULA HERE).

\dots

\section{Contributions and Evaluation}

%TODO
In regards to the contribution of this project, it has the intent to make it easier to produce content in LGP, and by doing so, promoting the inclusion and equality of opportunities for the deaf community.

\dots

\section{Conclusions and Future Work}

%TODO
This article presents a general view of the project that is currently in development.

\dots

The project development is not yet terminated, thus more functionalities are still going to be implemented in order to better fulfill the need of its target audience.
The main feature still to be implemented is the integration of the avatar that is responsible for translating the explanations displayed in plain text to LGP.
Another important feature not yet implemented is the displaying of the LGP readability score next to each explanation.
In regards to future work, the project can target an even broader audience by support other languages which, the previously mentioned avatar, is capable of performing the translation of plain text to the corresponding sign language.
In regards to future work, the API developed may be used to further enhance the functionalities of future or already existing GILT projects.

\dots

%
% ---- Bibliography ----
%
% BibTeX users should specify bibliography style 'splncs04'.
% References will then be sorted and formatted in the correct style.
%
% \bibliographystyle{splncs04}
% \bibliography{mybibliography}
%
\begin{thebibliography}{8}
    \bibitem{novak_2012}
        Novak, J.: Learning, creating, and using knowledge: concept maps as facilitative tools in schools and corporations. 2nd edn. Routlege, New York (2012)

    \bibitem{who_2020}
        WHO, \url{https://www.who.int/news-room/fact-sheets/detail/deafness-and-hearing-loss}. Last accessed 19 Aug 2020

    \bibitem{sts_2020}
        Spread The Sign, \url{https://www.spreadthesign.com/}. Last accessed 26 Aug 2020

    \bibitem{infopedia_2020}
        Infopédia, \url{https://www.infopedia.pt/dicionarios/lingua-gestual}. Last accessed 26 Aug 2020

    \bibitem{ctilg_2020}
        Ctilg, \url{http://www.ctilg.pt/}. Last accessed 27 Aug 2020

    \bibitem{serviin_2020}
        Serviin, \url{http://www.portaldocidadaosurdo.pt/Serviin}. Last accessed 27 Aug 2020

    \bibitem{ptread_2019}
        Antunes, H., Lopes, C.: Analyzing the adequacy of readability indicators to a non-english language.
        In: International Conference of the Cross-Language Evaluation Forum for European Languages,
        pages 149--155. Springer, 2019.

        %\bibitem{ref_article1}
        %Author, F.: Article title. Journal \textbf{2}(5), 99--110 (2016)

        %\bibitem{ref_lncs1}
        %Author, F., Author, S.: Title of a proceedings paper. In: Editor,
        %F., Editor, S. (eds.) CONFERENCE 2016, LNCS, vol. 9999, pp. 1--13.
        %Springer, Heidelberg (2016). \doi{10.10007/1234567890}

        %\bibitem{ref_book1}
        %Author, F., Author, S., Author, T.: Book title. 2nd edn. Publisher,
        %Location (1999)

        %\bibitem{ref_proc1}
        %Author, A.-B.: Contribution title. In: 9th International Proceedings
        %on Proceedings, pp. 1--2. Publisher, Location (2010)

        %\bibitem{ref_url1}
        %LNCS Homepage, \url{http://www.springer.com/lncs}. Last accessed 4
        %Oct 2017
\end{thebibliography}
\end{document}
