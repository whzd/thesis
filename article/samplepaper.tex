% This is samplepaper.tex, a sample chapter demonstrating the
% LLNCS macro package for Springer Computer Science proceedings;
% Version 2.20 of 2017/10/04
%
\documentclass[runningheads]{llncs}
%
\usepackage{graphicx}
% Used for displaying a sample figure. If possible, figure files should
% be included in EPS format.
%
% If you use the hyperref package, please uncomment the following line
% to display URLs in blue roman font according to Springer's eBook style:
% \renewcommand\UrlFont{\color{blue}\rmfamily}

\begin{document}
%
\title{Automatic Concept Explanation for Deaf Users}
%
%\titlerunning{Abbreviated paper title}
% If the paper title is too long for the running head, you can set
% an abbreviated paper title here
%
\author{André Dias\inst{1} \and
Nuno Escudeiro\inst{2}}
%
\authorrunning{A. Dias, N. Escudeiro}
% First names are abbreviated in the running head.
% If there are more than two authors, 'et al.' is used.
%
\institute{Instituto Superior de Engenharia do Porto\\
\email{1140279@isep.ipp.pt}\and
Instituto Superior de Engenharia do Porto\\
\email{nfe@isep.ipp.pt}}
%
\maketitle              % typeset the header of the contribution
%
\begin{abstract}
    The abstract should briefly summarize the contents of the paper in
    150--250 words.

    \keywords{Sign language \and Text summarization \and Automatic concept explanation.}
\end{abstract}
%
%
%
\section{Introduction}

Learning is the act or process of acquiring knowledge, and it's a part of the human life from birth to death.

The traditional way of learning is known as rote learning and it is defined by the memorization of information based in repetition.

One alternative to rote learning is meaningful learning, also known as conceptual learning.
Meaningful learning focus on understanding new knowledge and how it is related to the previously obtained.

Novak \cite{novak_2012} wrote in is book that meaningful learning has three requirements :

\begin{itemize}
    \item Relevant prior knowledge - The learner must already process information that has some relation to the new information to be learned.
    \item Meaningful material - It is composed by significant concepts and relevant to other, already acquired, knowledge.
    \item The willing to learn meaningfully - The learner chooses to consciously relate new knowledge to relevant knowledge already obtained.
\end{itemize}

Concepts are abstract or generic ideas that have a common set of features that are shared across multiple situations and contexts.
They are the key aspect of this approach, that help humans understand and relate knowledge.

Concept explanation can take a fundamental part of the learning process, whether it takes place in an educational environment, a day-to-day or a professional context.
The importance of concept explanation is even greater when it comes to languages with a small lexicon, in particular sign languages.

According to the World Health Organization \cite{who_2020}, as of march 2018, there were 466 million people with disabling hearing loss, this represents 5\% of the world population.
Without being able to hear, deaf people are not able to communicate using sound therefor they must use sign languages.

In Portugal the language used by the deaf community is known as Língua Gestual Portuguesa (LGP).
The lexicon of LGP is composed by multiple signs where each once represents a word or an expression of the main language, Portuguese.

\section{Related Work}

As mentioned in the introduction there is a lack for a solution capable to translate concepts from Portuguese to LGP.
There are however some partial solutions.

This section will describe those partial solutions as well as the existing Portuguese readability metrics that were analyzed in order to create the LGP readability metric.

\subsection{Online Dictionaries}

One of the most known online dictionaries with LGP content is the SpreadTheSign \cite{sts_2020}.
In this dictionary a user can search for a word and if there is a previously recorded video translation for that word, in the site database, it will be displayed for the user.
The search results only display the video of a person performing the corresponding sign, and the possibility to look at the same word in another sign language.

Another online dictionary that provides content for the LGP users is the Infopédia \cite{infopedia_2020}.
Although this is a Portuguese dictionary, it also contains a section for searching words in LGP.
Here the search results, not only provide a video translation of the word, but also an explanation in Portuguese in how to reproduce the sign shown in the video.

Online dictionaries as a solution is limited by the database of prerecorded videos, and the LGP lexicon.
The latter is a problem because they focus on a direct translation, word to sign, instead of trying to translate the meaning of the word.

\subsection{Sign Language Interpreter}

In regard to the sign language interpreters in Portugal there is CTILG \cite{ctilg_2020}, a company that provides professional LGP translation services in workshops, classes, congresses, events and more.
This company is responsible for the live translation of some morning TV shows.

A more affordable solution for a regular LGP user is the Serviin \cite{serviin_2020} which is a service that provides a interpreter to work as a middle-man between a deaf person and a targeted service/company.
This solution is available as a mobile app, with a very low cost for the deaf user, or through a  web app that is free.

This solution is limited by the interpreters own knowledge and their cost.
Also by utilizing this solution the LGP user is sacrificing some of his autonomy.

\subsection{Portuguese Readability Metrics}

Readability metrics are used to calculate a score that relates to the level of education a reader will need to fully understand the context of a given text.
There are some widely known and used metrics for the English language, such as Flesh Reading Ease, New Dale-Chall, SMOG, Flesh-Kincaid, Gunning Fog and so on.

In 2019, Antunes, H and Lopes, C published an article \cite{ptread_2019} that adapted the values of the metrics used to calculate the readability of text in English so it could be applied to Portuguese.
The adapted readability metrics are presented in table \ref{table:ptformulas}.

\begin{table}
    \caption{Adjusted Portuguese formulas.}
    \label{table:ptformulas}
    \begin{tabular}{l|l}
        \hline
        {} & {\bfseries Formula} \\
        \hline
        SMOG & \(16.830 \times \sqrt{CW \times 30 \div SE} - 23.809\)  \\
        \hline
        Flesch-Kincaid & \(0.883 \times WO \div SE + 17.347 \times SY \div WO - 41.239\) \\
        \hline
        ARI & \(6.286 \times CH \div WO + 0.927 \times WO \div SE - 36.551\) \\
        \hline
        Coleman Liau & \(5.730 \times CH \div WO - 171.365 \times SE \div WO - 6.662\) \\
        \hline
        Gunning Fog & \(0.760 \times WO \div SE + 58.600 \times CW \div WO - 12.166\) \\
        \hline
        \multicolumn{2}{l}{CH - characters, CW - complex words, SY - syllables, WO - words, SE - sentences}
    \end{tabular}
\end{table}

\section{Automatic Concept Explanation}

There are numerous words in Portuguese that don't have a sign that translates them.
Those words/expressions have to be explained using the available lexicon.

This problem is recurrent when it comes to scientific domains.
One example of this, is a concept like 'Nanotechnology' which doesn't have a sign for it, and so it's necessary to explain it using other words.

Automatic concept explanation was the solution design to aid in the process of explaining a given concept to the users of the LGP.

A user interacts with the solution through a web application, developed in React, where he can search for a given word or expression.

This input is sent to an Application Programming Interface (API), developed in Python, that is capable of generating explanations using Text Mining, Information Scraping and Information Retrieval techniques.

The generated explanations after being sorted are presented to the user in plain text.

\subsection{LGP Readability}

The developed API generates multiple explanations and even more for words that have different meanings based on the context.
This explanations need to be sorted in order to present the user with the ones best suited for their needs.

For a regular language, there are metrics that can be used to classify each expression in order to sort them accordingly.
One metric that can be used for this is the readability score.
This score indicates how easily a reader would understand the explanation that he's reading.

There are multiple formulas that could be used for calculating this score in Portuguese.
However there is none for the LGP.

After analyzing the existing formulas for calculating the readability for the PL, a new formula was developed to calculate the readability of LGP content.
The formula aims to calculate the readability of each sign using the following variables:

\begin{itemize}
    \item Number of hand configuration.
    \item Number of moments
    \item Number of hands used
    \item Number of facial expression used
\end{itemize}

The resulting formula is (INSERT FORMULA HERE).

\dots

\section{Contributions and Evaluation}

\dots

\section{Conclusions and Future Work}

The project development is not yet terminated, thus more functionalities are still going to be implemented in order to better fulfill the need of its target audience.

With the needs of the main users in mind the explanations displayed in plain text will also be translated to LGP with the help of an avatar.
The avatar responsible for the translation is a project already being developed by GILT, that will be integrated with the web application.

The API developed may be used to further enhance the functionalities of future or already existing GILT projects.

To target an even broader audience the project will support other languages which the previously mentioned avatar is capable of performing the translation of plain text to the corresponding sign language.

This article presents a general view of the project that is currently in development with the intent to make it easier to produce content in LGP.
By doing so, it will promote the inclusion and equality of opportunities for the deaf community.

\dots

%
% ---- Bibliography ----
%
% BibTeX users should specify bibliography style 'splncs04'.
% References will then be sorted and formatted in the correct style.
%
% \bibliographystyle{splncs04}
% \bibliography{mybibliography}
%
\begin{thebibliography}{8}
    \bibitem{novak_2012}
        Novak, J.: Learning, creating, and using knowledge: concept maps as facilitative tools in schools and corporations. 2nd edn. Routlege, New York (2012)

    \bibitem{who_2020}
        WHO, \url{https://www.who.int/news-room/fact-sheets/detail/deafness-and-hearing-loss}. Last accessed 19 Aug 2020

    \bibitem{sts_2020}
        Spread The Sign, \url{https://www.spreadthesign.com/}. Last accessed 26 Aug 2020

    \bibitem{infopedia_2020}
        Infopédia, \url{https://www.infopedia.pt/dicionarios/lingua-gestual}. Last accessed 26 Aug 2020

    \bibitem{ctilg_2020}
        Ctilg, \url{http://www.ctilg.pt/}. Last accessed 27 Aug 2020

    \bibitem{serviin_2020}
        Serviin, \url{http://www.portaldocidadaosurdo.pt/Serviin}. Last accessed 27 Aug 2020

    \bibitem{ptread_2019}
        Antunes, H., Lopes, C.: Analyzing the adequacy of readability indicators to a non-english language.
        In: International Conference of the Cross-Language Evaluation Forum for European Languages,
        pages 149--155. Springer, 2019.

        %\bibitem{ref_article1}
        %Author, F.: Article title. Journal \textbf{2}(5), 99--110 (2016)

        %\bibitem{ref_lncs1}
        %Author, F., Author, S.: Title of a proceedings paper. In: Editor,
        %F., Editor, S. (eds.) CONFERENCE 2016, LNCS, vol. 9999, pp. 1--13.
        %Springer, Heidelberg (2016). \doi{10.10007/1234567890}

        %\bibitem{ref_book1}
        %Author, F., Author, S., Author, T.: Book title. 2nd edn. Publisher,
        %Location (1999)

        %\bibitem{ref_proc1}
        %Author, A.-B.: Contribution title. In: 9th International Proceedings
        %on Proceedings, pp. 1--2. Publisher, Location (2010)

        %\bibitem{ref_url1}
        %LNCS Homepage, \url{http://www.springer.com/lncs}. Last accessed 4
        %Oct 2017
\end{thebibliography}
\end{document}
