% This is samplepaper.tex, a sample chapter demonstrating the
% LLNCS macro package for Springer Computer Science proceedings;
% Version 2.20 of 2017/10/04
%
\documentclass[runningheads]{llncs}
%
\usepackage{graphicx}
% Used for displaying a sample figure. If possible, figure files should
% be included in EPS format.
%
% If you use the hyperref package, please uncomment the following line
% to display URLs in blue roman font according to Springer's eBook style:
% \renewcommand\UrlFont{\color{blue}\rmfamily}

\begin{document}
%
\title{Automatic Concept Explanation}
%
%\titlerunning{Abbreviated paper title}
% If the paper title is too long for the running head, you can set
% an abbreviated paper title here
%
\author{André Dias\inst{1} \and
Nuno Escudeiro\inst{2}}
%
\authorrunning{A. Dias, N. Escudeiro}
% First names are abbreviated in the running head.
% If there are more than two authors, 'et al.' is used.
%
\institute{Instituto Superior de Engenharia do Porto\\
\email{1140279@isep.ipp.pt}\and
Instituto Superior de Engenharia do Porto\\
\email{nfe@isep.ipp.pt}}
%
\maketitle              % typeset the header of the contribution
%
\begin{abstract}
The abstract should briefly summarize the contents of the paper in
150--250 words.

\keywords{First keyword  \and Second keyword \and Another keyword.}
\end{abstract}
%
%
%
\section{Introduction}

Learning is the act or process of acquiring knowledge, and it's a part of the human life from birth to death.

\dots

Concepts are abstract or generic ideas that have a common set of features that are shared across multiple situations and contexts.

\dots

\section{Automatic Concept Explanation}

Concept explanation can take a fundamental part of the learning process, whether it takes place in an educational environment, a day-to-day or a professional context.
The importance of concept explanation is even greater when it comes to languages with a small lexicon, in particular sign languages.

\dots

\section{Future Development}

\dots

\section{Conclusion}

\dots

%
% ---- Bibliography ----
%
% BibTeX users should specify bibliography style 'splncs04'.
% References will then be sorted and formatted in the correct style.
%
% \bibliographystyle{splncs04}
% \bibliography{mybibliography}
%
\begin{thebibliography}{8}
\bibitem{ref_article1}
Author, F.: Article title. Journal \textbf{2}(5), 99--110 (2016)

\bibitem{ref_lncs1}
Author, F., Author, S.: Title of a proceedings paper. In: Editor,
F., Editor, S. (eds.) CONFERENCE 2016, LNCS, vol. 9999, pp. 1--13.
Springer, Heidelberg (2016). \doi{10.10007/1234567890}

\bibitem{ref_book1}
Author, F., Author, S., Author, T.: Book title. 2nd edn. Publisher,
Location (1999)

\bibitem{ref_proc1}
Author, A.-B.: Contribution title. In: 9th International Proceedings
on Proceedings, pp. 1--2. Publisher, Location (2010)

\bibitem{ref_url1}
LNCS Homepage, \url{http://www.springer.com/lncs}. Last accessed 4
Oct 2017
\end{thebibliography}
\end{document}
