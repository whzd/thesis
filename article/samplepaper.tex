% This is samplepaper.tex, a sample chapter demonstrating the
% LLNCS macro package for Springer Computer Science proceedings;
% Version 2.20 of 2017/10/04
%
\documentclass[runningheads]{llncs}
%
\usepackage{graphicx}
% Used for displaying a sample figure. If possible, figure files should
% be included in EPS format.
%
% If you use the hyperref package, please uncomment the following line
% to display URLs in blue roman font according to Springer's eBook style:
% \renewcommand\UrlFont{\color{blue}\rmfamily}

\begin{document}
%
\title{Automatic Concept Explanation}
%
%\titlerunning{Abbreviated paper title}
% If the paper title is too long for the running head, you can set
% an abbreviated paper title here
%
\author{André Dias\inst{1} \and
Nuno Escudeiro\inst{2}}
%
\authorrunning{A. Dias, N. Escudeiro}
% First names are abbreviated in the running head.
% If there are more than two authors, 'et al.' is used.
%
\institute{Instituto Superior de Engenharia do Porto\\
\email{1140279@isep.ipp.pt}\and
Instituto Superior de Engenharia do Porto\\
\email{nfe@isep.ipp.pt}}
%
\maketitle              % typeset the header of the contribution
%
\begin{abstract}
The abstract should briefly summarize the contents of the paper in
150--250 words.

\keywords{First keyword  \and Second keyword \and Another keyword.}
\end{abstract}
%
%
%
\section{Introduction}

Learning is the act or process of acquiring knowledge, and it's a part of the human life from birth to death.

The traditional way of learning is known as rote learning and it is defined by the memorization of information based in repetition.

One alternative to rote learning is meaningful learning, also known as conceptual learning.
Meaningful learning focus on understanding new knowledge and how it is related to the previously obtained.

Novak \cite{novak_2012} wrote in is book that meaningful learning has three requirements :

\begin{itemize}
    \item Relevant prior knowledge - The learner must already process information that has some relation to the new information to be learned.
    \item Meaningful material - It is composed by significant concepts and relevant to other, already acquired, knowledge.
    \item The willing to learn meaningfully - The learner chooses to consciously relate new knowledge to relevant knowledge already obtained.
\end{itemize}

Concepts are abstract or generic ideas that have a common set of features that are shared across multiple situations and contexts.
They are the key aspect of this approach, that help humans understand and relate knowledge.

Concept explanation can take a fundamental part of the learning process, whether it takes place in an educational environment, a day-to-day or a professional context.
The importance of concept explanation is even greater when it comes to languages with a small lexicon, in particular sign languages.

According to the World Health Organization \cite{who_2020}, as of march 2018, there were 466 million people with disabling hearing loss, this represents 5\% of the world population.
Without being able to hear, deaf people are not able to communicate using sound therefor they must use sign languages.

In Portugal the language used by the deaf community is known as Portuguese Sign Language (PSL).
The lexicon of PSL is composed by multiple signs where each once represents a word or an expression of the Portuguese Language (PL).

\section{Automatic Concept Explanation}

There are numerous words in PL that don't have a sign that translates them.
Those words/expressions have to be explained using the available lexicon.

This problem is recurrent when it comes to scientific domains.
One example of this, is a concept like 'Nanotechnology' which doesn't have a sign for it, and so it's necessary to explain it using other words.

Automatic concept explanation was the solution design to aid in the process of explaining a given concept to the users of the PSL.

A user interacts with the solution through a web application, developed in React, where he can search for a given word or expression.

This input is sent to an Application Programming Interface (API), developed in Python, that is capable of generating explanations using Text Mining, Information Scraping and Information Retrieval techniques.

The generated explanations after being sorted are presented to the user in plain text.

\subsection{Portuguese Sign Language Readability}

The developed API generates multiple explanations and even more for words that have different meanings based on the context.
This explanations need to be sorted in order to present the user with the ones best suited for their needs.

For a regular language, there are metrics that can be used to classify each expression in order to sort them accordingly.
One metric that can be used for this is the readability score.
This score indicates how easily a reader would understand the explanation that he's reading.

There are multiple formulas that could be used for calculating this score for the PL.
However there is none for the PSL.

After analyzing the existing formulas for calculating the readability for the PL, a new formula was developed to calculate the readability of PSL content.
The formula aims to calculate the readability of each sign using the following variables:

\begin{itemize}
    \item Number of hand configuration.
    \item Number of moments
    \item Number of hands used
    \item Number of facial expression used
\end{itemize}

The resulting formula is (INSERT FORMULA HERE).

\dots

\section{Future Development}

The project development is not yet terminated, thus more functionalities are still going to be implemented in order to better fulfill the need of its target audience.

With the needs of the main users in mind the explanations displayed in plain text will also be translated to PSL with the help of an avatar.
The avatar responsible for the translation is a project already being developed by GILT, that will be integrated with the web application.

The API developed may be used to further enhance the functionalities of future or already existing GILT projects.

To target an even broader audience the project will support other languages which the previously mentioned avatar is capable of performing the translation of plain text to the corresponding sign language.

\dots

\section{Conclusion}

This article presents a general view of the project that is currently in development with the intent to make it easier to produce content in PSL.
By doing so, it will promote the inclusion and equality of opportunities for the deaf community.

\dots

%
% ---- Bibliography ----
%
% BibTeX users should specify bibliography style 'splncs04'.
% References will then be sorted and formatted in the correct style.
%
% \bibliographystyle{splncs04}
% \bibliography{mybibliography}
%
\begin{thebibliography}{8}
\bibitem{novak_2012}
Novak, J.: Learning, creating, and using knowledge: concept maps as facilitative tools in schools and corporations. 2nd edn. Routlege, New York (2012)

\bibitem{who_2020}
WHO Deafness and hearing loss, \url{https://www.who.int/news-room/fact-sheets/detail/deafness-and-hearing-loss}. Last accessed 19 Aug 2020

%\bibitem{ref_article1}
%Author, F.: Article title. Journal \textbf{2}(5), 99--110 (2016)

%\bibitem{ref_lncs1}
%Author, F., Author, S.: Title of a proceedings paper. In: Editor,
%F., Editor, S. (eds.) CONFERENCE 2016, LNCS, vol. 9999, pp. 1--13.
%Springer, Heidelberg (2016). \doi{10.10007/1234567890}

%\bibitem{ref_book1}
%Author, F., Author, S., Author, T.: Book title. 2nd edn. Publisher,
%Location (1999)

%\bibitem{ref_proc1}
%Author, A.-B.: Contribution title. In: 9th International Proceedings
%on Proceedings, pp. 1--2. Publisher, Location (2010)

%\bibitem{ref_url1}
%LNCS Homepage, \url{http://www.springer.com/lncs}. Last accessed 4
%Oct 2017
\end{thebibliography}
\end{document}
