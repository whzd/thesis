% Chapter 2

\chapter{Context and the State of Art}
\label{chap:Chapter2}

This chapter presents a review of the most relevant topics (e.g. Information Retrival, Information Extraction) and similar approaches to the problem to be solved.

\section{Information Retrival}


Information Retrival (IR), as the name suggests, is the act of retriving information from a source.
But this definition can be very broad.
Christopher D. Manning et al. (\citeyear{Reference1}) wrote on his book that Information Retrival is finding materials of an unstructured nature that satisfies an information need from within large collections.

\section{Information Extraction}

As society became more data oriented having access to both structured and unstructured data became easy.
The difference between those those types of data is that structured data is semantically defined for a target domain and is interpreted with respect to category and context.
Therefor the need for applications capable of extracting structured data had increased.

Information Extraction (IE) is the name given to the process of automatically extracting structured information from an unstructured sources, mainly texts.
The result of an IE process is different for every case since it can be tailored according to the application needs.
Nowadays this applications can be used to fullfill personal, scientific and enterprise needs.

With the evolution of technology, IE also evolved and different models for the extraction of information were created.
These models are the following: Rule-based, Statictical, Hybrids (both rule-based and statistical) and Conditional Random Fields.

One example of a rule-based approach is given by Rosa Del Gaudio and António Branco (\citeyear{Reference4}), in this paper the authors created a IE system that was capable extracting the definition of a word from texts written in Portuguese.

Another example \dots

\section{Related Work}

In this section are discussed studies that are related to the work here reported and whose results and methods may be put in contrast with those produced in this work.

After extensive research, this project seems to be the first to try to accomplish the goal of generating a definition to a given word or expression using Text Mining, Information Retrival and Information Extraction.
Thanapon Noraset et al. (\citeyear{Reference3}) used Deep Learning, more precisely Recurrent Neural Networks (RNN) that learn to generate a definition from a pre-defined database\dots

Ke Ni and William Yang Wang (\citeyear{Reference2}) also chose a Deep Learning approach, using RNN to generate a definition of slang words or expressions in twits.
In this approach, the RNN was trained using an online, user contributed, dictionary called Urban Dictionary.
