% Chapter 2

\chapter{State of Art}
\label{chap:Chapter2}

This chapter presents a review of the most relevant topics, namely Portuguese Sign Language, Information Retrieval, Information Extraction, similar approaches to the problem to be solved and technologies that can be useful.

\section{Portuguese Sign Language}

Sign language was created to allow people to communicate through signs instead of sounds.
This is particularly useful for those that have some hearing impairment that made them incapable of learning to communicate through sounds.
In the context of a sign language, a sign, which is composed by the movement of the upper limbs, the configuration and orientation of the hands and facial expression, is used to represent an idea.

According to the Portuguese Deaf Association there are around 150000 people with some type of hearing impairment and around 30000 of those that use the Portuguese Sign Language (PSL) \autocite{gaspar_2015}.
This language was approved by the Constitution of the Portuguese Republic, in 1997, and became one of the three official languages in Portugal.

There are three ways to structure a sentence In PSL: subject-object-verb, subject-verb-object or object-subject-verb.
The other parts used in Portuguese Language, like the propositions and the articles, are omitted when converted to \gls{PSL} \autocite{bento_2013}, \autocite{martins_2011}.

Some grammatical characteristics of the Portuguese Sign Language are \autocite{bento_2013}:

\begin{itemize}
    \item In most cases the prefix "women" is used to identify the female version of a being while the male version is identified by the lack of a prefix "male".
    \item To describe a quantity of a given subject a number can be added or the use of the suffix "many".
    \item To represent temporal placement its used the suffix "past" or "future" to the verb.
    \item The negation of a sentence is defined by the word "not" at the end.
    \item It is used an interrogative pronoun at the end of the sentence to represent it as a question.
\end{itemize}

\section{Information Retrieval}

The search for information is a human activity that was always present.
The World Wide Web brought the commodity of searching information from within one's home, where before it was required to go to a place that stored said information, mainly libraries.

Information Retrieval (IR), as the name suggests, is the act of retrieving information from a source, but this definition can be very broad.
\textcite{manning_2008} wrote on their book that Information Retrieval is finding materials of an unstructured nature that satisfies an information need from within large collections.

The IR systems can be arranged in three groups based on its scale.
This groups are: Web search, personal information retrieval and institutional, and domain-specific search.

\section{Information Extraction}

As society became more data oriented having access to both structured and unstructured data became easy.
The difference between those those types of data is that structured data is semantically defined for a target domain and is interpreted with respect to category and context.
Therefor the need for applications capable of extracting structured data had increased.

Information Extraction (IE) is the name given to the process of automatically extracting structured information from an unstructured sources, mainly texts.
The result of an IE process is different for every case since it can be tailored according to the application needs.
Nowadays this applications can be used to fulfill personal, scientific and enterprise needs.

With the evolution of technology, IE also evolved and different techniques for the extraction of information were developed.
This techniques are the following: Rule-based, Statistical, Hybrids (both rule-based and statistical) and Conditional Random Fields \autocite{sarawagi_2008}.

A rule-based approach was used by \textcite{gaudio_2007}.
In this paper the authors created a IE system that was capable extracting the definition of a word from texts written in Portuguese.

A statistical approach was used by \textcite{ventura_2014}.
In his PhD report, he presented an alternative approach to the extraction of relevant terms from text.
Since relevance of a term is not conceptual, the author proposes to extract all concepts, which have a less fuzzy nature, and let the downstream application decide the relevance of those concepts.
A concept in the text mining area consists of a word or sequence of words which possess semantic value.

\section{Related Work}

In this section are discussed studies that are related to the work here reported and whose results and methods may be put in contrast with those produced in this work.

After extensive research, this project seems to be the first to try to accomplish the goal of generating a definition to a given word or expression using Text Mining, Information Retrieval and Information Extraction.

Trying to achieve a similar goal using a different approach, \textcite{noraset_2016} used Deep Learning, more precisely \gls{RNN} to generate a definition g a definition for a given word and its embedding
The models were trained using a pre-defined database.

\textcite{ni_2017} also chose a Deep Learning approach, using \gls{RNN} to generate a definition of slang words or expressions in twits.
In this approach, the \gls{RNN} was trained using an online, user contributed, dictionary called Urban Dictionary.

In regards to translating PL to its design sing language PSL, there are some online dictionary capable of performing this task.
One of them is Spread The Sign, it allows the translation of 39 dialects into their respective sign language.
The translations are shown using a video recording of a person performing the signs that represent the searched term.

Another online sign language dictionary is the Sign BSL, this one only supports British English to British sign language.
Since this dictionary focus only on a single dialect each word has a lot more possible related meaning translations.
Once again the translation shown are pre recorded videos of a person performing the signs.

\section{Technology}

\subsection{NLTK}

\subsection{openNLP}

\subsection{Google Cloud NLP}

\subsection{Django}
