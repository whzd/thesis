% Chapter 2

\chapter{Context and the State of Art}
\label{chap:Chapter2}

This chapter presents a review of the most relevant topics (e.g. Information Retrival, Information Extraction) and similar approaches to the problem to be solved.

\section{Information Retrival}

\dots

As the name suggests, Information Retrival, is the act of retriving information from a source.
But this definition can be very broad.
Christopher D. Manning et al. (\citeyear{Reference4}) wrote on his book that Information Retrival is finding materials of an unstructured nature that satisfies an information need from within large collections.

\section{Information Extraction}

\dots

\section{Related Work}

The goal is to investigate approaches to the problem of generating a definition, or explanation to a given word or expression.

