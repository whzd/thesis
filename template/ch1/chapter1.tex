% Chapter 1
% 
\chapter{Introduction} % Main chapter title
\label{chap:Chapter1} % For referencing the chapter elsewhere, use Chapter~\ref{Chapter1}

\section{Context}

Concept explanation is an fundamental part of the learning process, \dots takes place in an educational environment, a day-to-day context or a profissional context.

\dots

\section{Problem}

The importance of concept explanation is even greater when it comes to languages with a small lexicon, in particular sign languages, namely the \gls{PSL}.
The lexicon of \gls{PSL} is composed by multiple signs where each once represents a word or an expression of the \gls{PL}.
However, there are numerous words in \gls{PL} that don't have a sign that translates them.
Those words/expressions have to be explained using the available lexicon.
This problem is recurrent when it comes to scientific domains.
One example of this, is the a concept like 'Nanotechnology' which doesn't have a sign for it, and so it's necessary to explain it using other words.

\section{Objectives}

Taking into consideration the problem previously explained, the main objective of this project is to develop an application capable of generating the explanation of a given word or expression.
To be able to produce said explanation the application will use Text Mining, Information Scraping and Information Retrival techniques.

The resulting explanation will be presented has text to the user and also translated to Portuguese Sign Langue with the help of an avatar that was previously developed in GILT-ISEP.

This project has the intent to make it easier to produce content in Portuguese Sign language, and by doing so, promoting the inclusion and equality of opportunities for the deaf community. 

