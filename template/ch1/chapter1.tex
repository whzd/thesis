% Chapter 1
% 
\chapter{Introduction} % Main chapter title
\label{chap:Chapter1} % For referencing the chapter elsewhere, use Chapter~\ref{Chapter1}

\section{Context}

Ways of learning: rote learning meaningfully learning

Meaningful learning \dots

Meaningful learning has three requirements (\cite{novak_2012}): 

\begin{itemize}
    \item Relevant prior knowledge - The learner must already process information that has some relation to the new information to be learned.
    \item Meaningful material - It is composed by significant concepts and relevant to other, already acquired, knowledge.
    \item The willing to learn meaningfully - The learner chooses to consciously relate new knowledge to relevant knowledge already obtained.
\end{itemize}

Concepts are abstract or generic ideas \dots
An example that helps understand this is the concept of chair.
A chair can take multiple shapes and forms but there are some characteristic that are share among all of them, such as having legs, a seat and a back rest.
Once you learn those, you understand the concept of it as a whole, and are able to identify most chairs as such.

\section{Problem}

Concept explanation is an fundamental part of the learning process, whether takes place in an educational environment, a day-to-day context or a profissional context.
The importance of concept explanation is even greater when it comes to languages with a small lexicon, in particular sign languages, namely the \gls{PSL}.
The lexicon of \gls{PSL} is composed by multiple signs where each once represents a word or an expression of the \gls{PL}.
However, there are numerous words in \gls{PL} that don't have a sign that translates them.
Those words/expressions have to be explained using the available lexicon.
This problem is recurrent when it comes to scientific domains.
One example of this, is the a concept like 'Nanotechnology' which doesn't have a sign for it, and so it's necessary to explain it using other words.

\section{Objectives}

Build and api \dots

Taking into consideration the problem previously explained, the main objective of this project is to develop an application capable of generating the explanation of a given word or expression.
To be able to produce said explanation the application will use Text Mining, Information Scraping and Information Retrival techniques.

The resulting explanation will be presented has text to the user and also translated to Portuguese Sign Langue with the help of an avatar that was previously developed in GILT-ISEP.

This project has the intent to make it easier to produce content in Portuguese Sign language, and by doing so, promoting the inclusion and equality of opportunities for the deaf community. 

