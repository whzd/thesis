% Chapter 1
% 
\chapter{Introduction} % Main chapter title
\label{chap:Chapter1} % For referencing the chapter elsewhere, use Chapter~\ref{Chapter1}

\section{Context}

Learning is the act or process of acquiring knowledge, and it's a part of the human life from birth to death.
This ability is shared by humans and animals and it's being partially implemented to machines with the help of machine learning.

The traditional way of learing is known as rote learning and it is defined by the memorization of information based in repetition.
It has the advantage of quickly developing basic knowledge, an example of this is teaching a kid the alphabet or the numbers.
A major disadvantage of this approach is not be able to be used to teach more complex or abstract subjects or subjects that can have different meaning based on context.
Explaning a thopic like "Love" is not possible with this methodology.

One alternative to rote learning is meaningful learning, also known as conceptual learning.
Meaningful learning focus on understanding new knowledge 
\dots

Novak (\citeyear{novak_2012}) wrote in is book that meaningful learning has three requirements : 

\begin{itemize}
    \item Relevant prior knowledge - The learner must already process information that has some relation to the new information to be learned.
    \item Meaningful material - It is composed by significant concepts and relevant to other, already acquired, knowledge.
    \item The willing to learn meaningfully - The learner chooses to consciously relate new knowledge to relevant knowledge already obtained.
\end{itemize}

Concepts are abstract or generic ideas that have a common set of features that are shared across multiple situations and contexts.
\dots

An example that helps understand this is the concept of chair.
A chair can take multiple shapes and forms but there are some characteristic that are share among all of them, such as having legs, a seat and a back rest.
Once you learn those, you understand the concept of it as a whole, and are able to identify most chairs as such.

\section{Problem}

Concept explanation can take a fundamental part of the learning process, whether it takes place in an educational environment, a day-to-day or a profissional context.
The importance of concept explanation is even greater when it comes to languages with a small lexicon, in particular sign languages, namely the \gls{PSL}.
The lexicon of \gls{PSL} is composed by multiple signs where each once represents a word or an expression of the \gls{PL}.
However, there are numerous words in \gls{PL} that don't have a sign that translates them.
Those words/expressions have to be explained using the available lexicon.

This problem is recurrent when it comes to scientific domains.
One example of this, is the a concept like 'Nanotechnology' which doesn't have a sign for it, and so it's necessary to explain it using other words.

\section{Objectives}

Build and api \dots

Taking into consideration the problem previously explained, the main objective of this project is to develop an application capable of generating the explanation of a given word or expression.
To be able to produce said explanation the application will use Text Mining, Information Scraping and Information Retrival techniques.

The resulting explanation will be presented has text to the user and also translated to Portuguese Sign Langue with the help of an avatar that was previously developed in GILT-ISEP.

This project has the intent to make it easier to produce content in Portuguese Sign language, and by doing so, promoting the inclusion and equality of opportunities for the deaf community. 

