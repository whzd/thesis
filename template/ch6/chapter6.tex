% Chapter 6

\chapter{Evaluation} % Main chapter title
\label{chap:Chapter6} 

\section{Hypothesis}

\dots

\section{Evaluation metrics}

The utilization of text mining algorithms requires specific metrics to evaluate its performance.
This metrics are: 

\begin{itemize}
    \item Accuracy - It is the fraction of the correct predictions made by the model.
    It can be calculated using the following formula:

    \begin{equation}
    Accuracy = \frac{True Positives + True Negatives}{Total number of predictions}
    \label{eqn:Accuracy}
    \end{equation}

    \item Precision - It is the fraction of positive identification among all the identifications made by the model.
    It can be calculated using the following formula:
    
    \begin{equation}
    Precision = \frac{True Positives}{True Positives + False Positives}
    \label{eqn:Precision}
    \end{equation}

    \item Recall - It is the fraction of positive identification among all the possible identification to be made.
    It can be calculated using the following formula:
    
    \begin{equation}
    Recall = \frac{True Positives}{True Positives + False Negatives}
    \label{eqn:Recall}
    \end{equation}

    \item F-measure - It is the harmonic mean between Recall and Precision, in order to give a better preception of both in a single value.
    It can be calculated using the following formula:
    
    \begin{equation}
    F-measure = \frac{2*{Precision}*{Recall}}{Precision + Recall}
    \label{eqn:F-measure}
    \end{equation}

\end{itemize}

\dots

\section{Methodologies}

(integration, functionality and system tests)
The solution aims to solve a problem shared by a member of the deaf portuguese community.
This community is the main target audience therefor, utilizing a satisfaction inquiry is mandatory to proper evaluate how fitting and helpfull the solution can be.  

\dots
