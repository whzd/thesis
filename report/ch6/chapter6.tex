% Chapter 6

\chapter{Evaluation} % Main chapter title
\label{chap:Chapter6}

This chapter presents the tests made to the developed solution.
Due to delays during the project, only some tests were implemented and the overall solution was not properly evaluated.
The following sections start by describing the software tests implemented.
After that, it presents the project hypothesis as well as the methodology that was supposed to be used to evaluate its validity.

\section{Software tests}

The technical quality of the solution is measure using tests, namely integration tests, functional tests and system tests.
Has the development got to a state where those tests are reliable, they were executed using specific tools.

Integration tests are used to detect defects in the interfaces and in the interactions between integrated components.
They are important to catch system level issues, such as broken database schema.
In this project they were created as parts of the solution were being connected such as the Web Application and the Explanation API, the Explanation API and the Web scrapper, etc.

Functional tests are used to verify if the functionality of the software is in compliance with the specifications.
After the first prototype was created it was deployed to allow for user interaction.
The project coordinator would periodically test the functionalities in order to give feedback regarding defects or improvements to be made.

System tests are used to ensure that the system is functioning properly when all features are bundled together.
In this project they were created to ensure that the final version was ready to be used by the target audience.
One example is to verify if the solution could handle malformed input string.

\section{Hypothesis}

A hypothesis was defined in order for the project to be able to help the creation of \gls{LGP} content, and by doing so promoting the inclusion and equality of opportunities for the deaf community.
This hypothesis consists in verifying if it is possible to utilize Text Mining\cite{gupta2009survey}, Information Scraping\cite{young2006method}\cite{rose2017automatically} and Information Retrieval\cite{baeza1999modern} techniques to generate the explanation of a given word or expression in \gls{LGP} that does not exist in its lexicon.

Since this is the main functionality of the developed application the hypothesis to be tested can be the following:
\begin{center}
    \emph{H0: User experience}

    \emph{H1: The survey possess an overall rate above or equal to 3}

    \emph{H2: The survey do not possess an overall rate above or equal to 3}
\end{center}

\section{Methodology}

After it has been defined, the hypothesis will be tested using evaluating methodologies in order to assess its validation.
This methodology was a survey were the focus points were the usability, performance and reliability regarding the experience with the developed solution.
The created survey is presented at the end of this document, in the Appendix~\ref{AppendixA}.

In order for the survey to obtain meaningful results it was to be completed specifically by the target audience, which are the \gls{LGP} users.
To do so, this survey was meant to be shared to the students of a \gls{LGP} school.

Since this was not possible, the solution was evaluated by people that were not familiar with the \gls{LGP}.

A total of eleven people took the survey.
The majority were bachelor degree students between the age of 20 and 40.

The questions that were taken in consideration were the following:
\begin{itemize}
    \item \textbf{Q1} - Is the information presented in an organized matter?
    \item \textbf{Q3} - Is the graphics of the application pleasant?
    \item \textbf{Q4} - Did the explanations presented correspond adequately to the search?
    \item \textbf{Q5} - Is the waiting time to obtain a result satisfactory?
    \item \textbf{Q7} - Is the additional information presented relevant?
    \item \textbf{Q8} - Was the option to display images related to the explanation helpful?
\end{itemize}

Since this is not the desired target audience, some questions were not taken in consideration such as:
\begin{itemize}
    \item \textbf{Q2} - Are the functionalities easy to use?
    \item \textbf{Q6} - Is the readability score displayed adequate?
    \item \textbf{Q9} - Would this be an application that you would use in the future?
    \item \textbf{Q10} - Would you recommend this application to others?
\end{itemize}

In the Table~\ref{tab:surveyscale} is shown the scale used for the answers to the survey questions.

\begin{table}[H]
    \caption{Survey Scale's Description.}
    \label{tab:surveyscale}
    \centering
    \begin{tabular}{m{4cm}|m{4cm}}
        \tabhead{Scale} & \tabhead{Description} \\
        \hline
        1 & Completely Disagree \\
        \hline
        2 & Disagree \\
        \hline
        3 & Agree \\
        \hline
        4 & Completely Agree \\
    \end{tabular}
\end{table}

In the following Table~\ref{tab:surveyanswer} the acquired data from the survey is summarized, in the form of response percentage for each of the questions that were considered.

\begin{table}[H]
    \caption{Survey Answers Response Percentage.}
    \label{tab:surveyanswer}
    \centering
    \begin{tabular}{m{2cm}|m{2cm}|m{2cm}|m{2cm}|m{2cm}|m{2cm}}
        \tabhead{Question} & \tabhead{1} & \tabhead{2} & \tabhead{3} & \tabhead{4} & \tabhead{Average} \\
        \hline
        Q1 & 0\% & 0\% & 27.3\% & 72.7\% & 3.8 \\
        \hline
        Q3 & 0\% & 18.2\% & 45.5\% & 36.4\% & 3.2 \\
        \hline
        Q4 & 0\% & 0\% & 9.1\% & 90.9\% & 3.9\\
        \hline
        Q5 & 0\% & 0\% & 9.1\% & 90.9\% & 3.9 \\
        \hline
        Q7 & 0\% & 0\% & 9.1\% & 90.9\% & 3.9 \\
        \hline
        Q8 & 0\% & 0\% & 9.1\% & 90.9\% & 3.9 \\
    \end{tabular}
\end{table}

Even though there were few answers to these questions, and this is not the desired audience, over 90\% of the distribution is a 4/4, which is very positive.
Also the worse classification of any response was a 2/4 in the question that presented the worse overall result, that was regarding the visual aspect of the application.
Plus the average classification for all the questions is approximately 3.8/4, which passes the defined hypothesis of being higher or equal than 3/4.
This is a clear indication that the overall application was very well received by the testing users.

