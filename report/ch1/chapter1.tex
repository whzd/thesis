% Chapter 1
%
\chapter{Introduction} % Main chapter title
\label{chap:Chapter1} % For referencing the chapter elsewhere, use Chapter~\ref{Chapter1}

This chapter is dedicated to the presentation and introduction of the present project.
Firstly, by contextualizing it, establishing the problem associated with it and the goals set to achieve.

\section{Context}

Learning is the act or process of acquiring knowledge, and it's a part of the human life from birth to death.
This ability is shared by humans and animals and it's being partially implemented to machines with the help of machine learning.
Humans learn by interaction with other others or the environment.

The traditional way of learning is known as rote learning and it is defined by the memorization of information based in repetition.
It has the advantage of quickly developing basic knowledge, an example of this is teaching a kid the alphabet or the numbers.
A major disadvantage of this approach is not be able to be used to teach more complex or abstract subjects or subjects that can have different meaning based on context.
Explaining a topic like "Love" is not possible with this methodology.

One alternative to rote learning is meaningful learning, also known as conceptual learning.
Meaningful learning focus on understanding new knowledge and how it is related to the previously obtained.
One disadvantage of this approach is that not every human has the same previous knowledge, therefore, if not tailored, it will take different people different time to be understand.
On the other hand, a major advantage is the encouragement of understanding which benefits retention of knowledge.

Novak wrote in is book\cite{novak2010learning} that meaningful learning has three requirements :

\begin{itemize}
    \item Relevant prior knowledge - The learner must already process information that has some relation to the new information to be learned.
    \item Meaningful material - It is composed by significant concepts and relevant to other, already acquired, knowledge.
    \item The willing to learn meaningfully - The learner chooses to consciously relate new knowledge to relevant knowledge already obtained.
\end{itemize}

Concepts are the key aspect of this approach, that help humans understand and relate knowledge.
Concepts are abstract or generic ideas that have a common set of features that are shared across multiple situations and contexts.

Humans have learned concepts even if they are not aware of it.
An example of it is shown every time they take a sit on chair.
Chairs can take very different designs but they all share some characteristic, such as having a mean of support, a seat and a back rest.
The concept of chair is made by those shared characteristic and once one understands that concept he will be able to identify most chairs as such.

\section{Problem}

Concept explanation can take a fundamental part of the learning process\cite{ghorbani2019towards}, whether it takes place in an educational environment, a day-to-day or a professional context.
The importance of concept explanation is even greater when it comes to languages with a small lexicon, in particular sign languages.

According to the World Health Organization \cite{who2018}, as of march 2018, there were 466 million people with disabling hearing loss, this represents 5\% of the world population.
Without being able to hear, deaf people are not able to communicate using sound therefor they must use sign languages.

The number of different sign languages are not precisely defined, but most countries have their one sign language.

In Portugal the language used by the deaf community is known as \gls{LGP}.
This language is also used by the surrounding community, such as their relatives, teachers, and so on.
The lexicon of \gls{LGP} is composed by multiple signs where each once represents a word or an expression in Portuguese.
However, the opposite is not true.
There are numerous words in Portuguese that don't have a sign that translates them.

In order for a \gls{LGP} user to understand those words/expressions the explanation must be formed by the available lexicon\cite{musselman2000children}, yet there are no \gls{LGP} encyclopedias.
The closest thing available are the \gls{LGP} dictionaries but they fail to translate words that have no sign that can directly translate them.

This problem is recurrent when it comes to scientific domains.
To understand a concept like 'Nanotechnology', which there are no signs for it, a \gls{LGP} user has to access tools with content made for the Portuguese users.

Even though \gls{LGP} is the language taught to the people that are not able to communicate using Portuguese, they are two very different languages.
Each with their own lexicon and grammar, so most of the \gls{LGP} users will struggle to understand Portuguese content.

A \gls{LGP} user might also resort to a sign language interpreter that would explain the meaning of the word to him, but this would be an even less practical solution

\section{Objectives}

Taking into consideration the problem previously explained, the solution consists of two major components.
An \gls{API} capable of generating explanations of a given word, using Text Mining, Information Scraping and Information Retrieval techniques.
And a web app for the users to interact with, that allows them to search for a word and displays the generated explanation in plain text.

To allow the translation from plain text to \gls{LGP}, the web app will be enhanced with an avatar that was previously developed by \gls{GILT}.

This project has the intent to make it easier to produce content in \gls{LGP}, and by doing so, promoting the inclusion and equality of opportunities for the deaf community.
