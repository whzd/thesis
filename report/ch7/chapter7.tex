% Chapter 7

\chapter{Conclusion} % Main chapter title
\label{chap:Chapter7}

The main goal of this project was to develop an application that was capable of aiding in the process of explaining a concept.
The key features was to use Information Retrieval, Information Extraction and Text Mining techniques to accomplish this goal.

After contextualizing the problem and introducing the problem and goals, the state of art was created.
It started by explaining what is the \gls{LGP} by approaching its history, the signs composition, its grammatical characteristics and how concept explanation is important.
Following this it presented what is Information Retrieval, Information Extraction and Text Mining by explaining their functionalities.
Later the two categories of partial solution that are currently available were presented.
It was concluded that solutions available were lacking areas that can be fulfilled by the current project, and that there's a hole in the market for what this project is trying to achieve, in terms of being practical, complete and easily accessible.
The readability metrics available were also thoroughly analyzed since the formula created in this project has an very important role in the explanations that are displayed.
After that, the analysis of some developed solutions that tried to achieve a similar goals were presented.
Lastly, the technologies that were mentioned throughout the project were also explained.

A value analysis was produced that contextualized the project in its scope, presenting the model responsible for identifying opportunities and selecting new ideas and exhibiting a value proposition.
To better structure the various business characteristics a business model CANVAS was also created.
During the value analysis, \gls{AHP} was used to help in the process of deciding which natural language processing tools to use when implementing the Text Mining techniques.
NLTK revealed to be the best choice based on the parameters that were taken in consideration.

During the design process, the requirements were engineered, in order for the system to be able to hold the envisioned functionalities.
This was accomplished through the definition of use cases and a set of non-functional requirements.
Moreover, the project architecture was defined using a logic view, a process view, the design of the functional requirements that were explain in detail and a deployment view.

After designing the requirements and defining the architecture, the specified functionalities were implemented.
During the implementation of the solution some changes had to be made.
The reasoning as well as the impact of this changes was thoroughly explained and the diagrams created during the design were altered to represent the final solution.

As the implementation progressed, the system was tested using integration tests, functional tests and system tests.

To evaluate the final solution a survey was created.
It was intended for it to be fulfilled by the target audience, but due to some delays this was not possible.
Instead it was people that were not familiar with \gls{LGP} that took the survey.
This was not ideal since some of the features of the solution, like the readability score weren't able to be properly evaluated.

\section{Accomplished Goals}

From the defined objectives, it can be said that the main goal, relative to the generation of explanations using Information Retrieval, Information Extraction and Text mining techniques, was achieved.

In regards to the objectives presented in the beginning of this document, the completed objectives are the following:
\begin{itemize}
        \item Develop an API capable of generating explanation of a given word using the upper mentioned techniques.
        \item Develop a Web Application for the users to interact with the developed API.
\end{itemize}

In regard to the integration of the Avatar with the developed Web Application, it considered not as important as the other objectives and due to some delays it was not implemented.

\section{Limitations and Future Work}

The biggest limitation, as previously mentioned, was the delays.
They occur due COVID-19 and some other health related problem.
This limitation had an impact in the developed solution as well as in the evaluation of the solution that was not able to be accomplished as it was intended.

In regards to the future work, there are some tasks that can improve the overall quality of the solution.
The first and most obvious one is the integration of the Avatar technology which would help in the translation process.
The second tasks is to improve the second approach by storing pages that were already crawled in order to reduce the time required to generate explanation.
The second approach can be further improved by implementing a way to better identify context difference.
Lastly, an evaluation made by the target audience since it's the only way to truly understand if this solution can target their special needs.

\section{Final Appreciation}

It was understood from the start that this dissertation would be a challenge.
Given that I had no previous knowledge of none of the proposed techniques to be used nor about \gls{LGP}.

From a professional standpoint, having to consider the special needs of some users will have and impact in future projects in order to make them accessible to as much people as possible.
Also the knowledge obtain in regards to web crawling, web scraping and text summarization will definitely have an impact in some personal project I had planned to develop.

In a more personal level, the development of this dissertation made me improve in my time management skills because I had recently started working and had to keep a healthy balance between both.

To summarize, this project global appreciation is positive.
However, there is a sentiment that under normal circumstances, the final product would be much better.
