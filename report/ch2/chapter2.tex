% Chapter 2

\chapter{State of Art}
\label{chap:Chapter2}

This chapter presents a review of the most relevant topics, namely Portuguese Sign Language, Information Retrieval, Information Extraction, similar approaches to the problem to be solved and technologies that can be useful.

\section{Língua Gestual Portuguesa}

Sign language was created to allow people to communicate through signs instead of sounds.
This is particularly useful for those that have some hearing impairment that made them incapable of learning to communicate through sounds.

The \gls{LGP} has its origins in the first Portuguese school for the deaf that was created in Lisbon by a Swedish educator by the name Pär Aron Borg.
This educator introduced an adaptation of the Swedish manual alphabet that was used by the deaf community to communicate.
Even though, currently, there are no similarities in the vocabulary of the \gls{LGP} and the Swedish sign language, the alphabet still shows the common origin\cite{oliveira2013tradutor}\cite{escudeiro2015virtual}

According to the Portuguese Deaf Association there are around 150000 people with some type of hearing impairment and around 30000 of those that use the \gls{LGP}\cite{gaspar2015if2lgp}.
This language was approved by the Constitution of the Portuguese Republic, in 1997, and became one of the three official languages in Portugal.

In the context of a sign language, a sign, is used to represent an idea and it is composed by the movement and position of the upper limbs.
Paula Escudeiro et al.\cite{escudeiro2015virtual} present the components of a \gls{LGP} sign as manual and non-manual.

The manual component consists in every variables related to hands which includes:
\begin{itemize}
        \item Configuration of the hand - The form that each hand takes while executing a gesture.
        \item Orientation of the palm of the hand - Some configurations only differ in the palm's orientation.
        \item Location of articulation - The area where the gesture is performed (Connected to a body part, touching a body part or just in front of the user).
        \item Movement of the hand - The motion of the hands during the execution of a gesture.
\end{itemize}

The non-manual component consists in the other variables that take part when representing a sign, which includes:
\begin{itemize}
        \item Body movement - The leaning of the torso which represents a temporal context.
        \item Facial expressions - Used to add a sense of emotion to the speech.
\end{itemize}

There are three ways to structure a sentence In \gls{LGP}: SOV (subject-object-verb), SVO (subject-verb-object) or OSV (object-subject-verb).
The predominate sequence used is the SOV\cite{sousa2012interpretaccao}\cite{correia2015linguas} but it's entirely up to the user to chose the structure to use.
Since there are no rule for this, the same user may choose to use a different structure for different sentences\cite{martins2011letra}.
The other parts used to construct a sentence in Portuguese, like the propositions and the articles, are omitted when converted to \gls{LGP}\cite{bento2014avatares}.

Some grammatical characteristics of the Portuguese Sign Language are\cite{bento2014avatares}:
\begin{itemize}
    \item In most cases the prefix "women" is used to identify the female version of a being while the male version is identified by the lack of a prefix "male".
    \item To describe a quantity of a given subject a number can be added or the use of the suffix "many".
    \item To represent temporal placement its used the suffix "past" or "future" to the verb.
    \item The negation of a sentence is defined by the word "not" at the end.
    \item It is used an interrogative pronoun at the end of the sentence to represent it as a question.
\end{itemize}

As already mentioned in the introduction the \gls{LGP} lexicon is quite small in comparison to Portuguese.
So many Portuguese words are represented by a logical decomposition of its meaning.
Using the word "Laranjeira" as an example, this the Portuguese word for orange tree.
Since there are no sign for this word, it is represented by the signs "Árvore" and "Laranja" (Tree and Orange respectively).

However when a \gls{LGP} user is faced with a word that he does not know its meaning he will spell each letter of the word.
For short words this can be a practical solution but for long words like "Nanotecnologia" not so much.
Spelling this word would require 13 sings while representing its meaning would take around 7 (e.g. Ciência, Estudar, Manipular, Materia, Tamanho, Muito, Pequeno).

\section{Text Mining}

\dots

\section{Information Retrieval}

The search for information is a human activity that was always present.
The World Wide Web brought the commodity of searching information from within one's home, where before it was required to go to a place that stored said information, mainly libraries.

Information Retrieval (IR), as the name suggests, is the act of retrieving information from a source, but this definition can be very broad.
Manning et al.\cite{manning2008introduction} wrote on their book that Information Retrieval is finding materials of an unstructured nature that satisfies an information need from within large collections.

\dots %TODO

The IR systems can be arranged in three groups based on its scale.
This groups are: Web search, personal information retrieval and institutional, and domain-specific search.

\section{Information Extraction}

As society became more data oriented having access to both structured and unstructured data became easy.
The difference between those those types of data is that structured data is semantically defined for a target domain and is interpreted with respect to category and context.
Therefor the need for applications capable of extracting structured data had increased.

Information Extraction (IE) is the name given to the process of automatically extracting structured information from an unstructured sources, mainly texts.
The result of an IE process is different for every case since it can be tailored according to the application needs.
Nowadays this applications can be used to fulfill personal, scientific and enterprise needs.

With the evolution of technology, IE also evolved and different techniques for the extraction of information were developed.
This techniques are the following: Rule-based, Statistical, Hybrids (both rule-based and statistical) and Conditional Random Fields\cite{sarawagi2008information}.

A rule-based approach was used by Del Gaudio et al.\cite{del2007automatic}.
In this paper the authors created a IE system that was capable extracting the definition of a word from texts written in Portuguese.

A statistical approach was used by Ventura\cite{ventura2014automatic}.
In his PhD report, he presented an alternative approach to the extraction of relevant terms from text.
Since relevance of a term is not conceptual, the author proposes to extract all concepts, which have a less fuzzy nature, and let the downstream application decide the relevance of those concepts.
A concept in the text mining area consists of a word or sequence of words which possess semantic value.

\subsection{Online Dictionaries}

One of the most known online dictionaries with \gls{LGP} content is the Spreadthesign\footnote{https://www.spreadthesign.com/}.
In this dictionary an user can search for a word and if there's a previously recorded video translation for that word, at the site database, it will be displayed for the user.
The search results only display the video of a person performing the corresponding sign, and the possibility to look at the same word in another sign language.

Another online dictionary that provides content for the \gls{LGP} users is the Infopédia\footnote{https://www.infopedia.pt/dicionarios/lingua-gestual}.
Although this is a Portuguese dictionary, it also contains a section for searching words in \gls{LGP}.
Here, the search results, not only provide a video translation of the word, but also an explanation in Portuguese on how to reproduce the sign shown in the video.

Online dictionaries as a solution, is limited by the database of prerecorded videos, and the \gls{LGP} lexicon.
The latter is a problem, because they focus on a direct translation, word to sign, instead of trying to translate the meaning of the word.

\subsection{Sign Language Interpreter}

In regard to the sign language interpreters in Portugal, there is CTILG\footnote{http://www.ctilg.pt/}, a company that provides professional \gls{LGP} translation services in workshops, classes, congresses, events and more.
This company is responsible for the live translation of some morning TV shows.

A more affordable solution for a regular \gls{LGP} user, is the Serviin\footnote{http://www.portaldocidadaosurdo.pt/Serviin} which is a service that provides an interpreter to work as a middle-man between a deaf person and a targeted service/company.
This solution is available as a mobile app, with a very low cost for the deaf user, or through a  web app that is free.

This solution is limited by the interpreters own knowledge and their cost.
Also by utilizing this solution, the \gls{LGP} user is sacrificing some of his autonomy.

\subsection{Readability Metrics}

Readability metrics are used to calculate a score\cite{meyer2003text}, that relates to the level of education a reader will need, to fully understand the context of a given text.
There are some widely known and used metrics for the English language, such as Flesh Reading Ease, New Dale-Chall, SMOG, Flesh-Kincaid, Gunning Fog and so on.

\dots %TODO

\subsection{Portuguese Readability Metrics}

In 2019, Antunes et. al.\cite{antunes2019analyzing} published an article that adapted the values of the metrics used to calculate the readability of text in English so it could be applied to Portuguese.
The adapted readability metrics are presented in Table \ref{table:ptformulas}.

\begin{table}
    \caption{Adjusted Portuguese formulas.}
    \label{table:ptformulas}
    \begin{tabular}{l|l}
        \hline
        {} & {\bfseries Formula} \\
        \hline
        SMOG & \(16.830 \times \sqrt{CW \times 30 \div SE} - 23.809\)  \\
        \hline
        Flesch-Kincaid & \(0.883 \times WO \div SE + 17.347 \times SY \div WO - 41.239\) \\
        \hline
        ARI & \(6.286 \times CH \div WO + 0.927 \times WO \div SE - 36.551\) \\
        \hline
        Coleman Liau & \(5.730 \times CH \div WO - 171.365 \times SE \div WO - 6.662\) \\
        \hline
        Gunning Fog & \(0.760 \times WO \div SE + 58.600 \times CW \div WO - 12.166\) \\
        \hline
        \multicolumn{2}{l}{CH - characters, CW - complex words, SY - syllables, WO - words, SE - sentences}
    \end{tabular}
\end{table}

\section{Related Work}

After extensive research, this project seems to be the first to try to accomplish the goal of generating a definition to a given word or expression using Text Mining, Information Retrieval and Information Extraction.

Trying to achieve a similar goal using a different approach, Noraset Thanapon et al.\cite{noraset2016definition} chose a Deep Learning approach, that used a \gls{RNN} model that used distributed representations of words, also known as word embeddings, to generate dictionary representations.
The models were trained using a pre-defined data set.

Ni Ke et al.\cite{ni2017learning} also chose a Deep Learning approach and a \gls{RNN} model to generate a explanation of a given word or expression from "twits" which is the name given to the posts made by the users of Twitter\footnote{https://twitter.com/}.
The focus was non-standard expressions, such as slang, presented in this posts.
The \gls{RNN} was trained using an online, user contributed, dictionary called Urban Dictionary\footnote{https://www.urbandictionary.com/).

\dots %TODO

\section{Technology}

\dots

\subsection{NLTK}

\dots

\subsection{openNLP}

\dots

\subsection{Google Cloud NLP}

\dots

\subsection{Flask}

\dots

\subsection{React}

\dots

\subsection{Scrapy}

\dots

\subsection{Beautiful Soup 4 (bs4)}

\dots

\subsection{Virtual Sign Avatar}

This avatar was a project developed by GILT (Games, Interaction and Learning Technologies) and is capable of translate Portuguese text to \gls{LGP}.

The visual part of the avatar as well as the animations it performs where created using Blender\footnote{https://www.blender.org/}.
Its appearance  was made identical to the human body to help the process of translation\cite{escudeiro2015virtual}.

All the animations available to be performed by the avatar during the translation are stored in a database that relates each animation to the respective text.
When is requested the translation of a text that is not present in the database the avatar will perform the animation of each letter of that text.
