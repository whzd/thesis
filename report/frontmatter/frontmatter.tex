% we include the glossary here (frontmatter is included with \input, so this command is as if it was in main.tex)
%All acronyms must be written in this file.
\newacronym{PSL}{PSL}{Portuguese Sign Language}
\newacronym{PL}{PL}{Portuguese Language}
\newacronym{API}{API}{Application Programming Interface}
\newacronym{MVP}{MVP}{Mininum Value Product}
\newacronym{DB}{DB}{Database}
\newacronym{NCD}{NCD}{New Concept Development Model}
\newacronym{CI}{CI}{Consistency Index}
\newacronym{CR}{CR}{Consistency Ratio}
\newacronym{RI}{RI}{Random Consistency Index}

\frontmatter % Use roman page numbering style (i, ii, iii, iv...) for the pre-content pages

\pagestyle{plain} % Default to the plain heading style until the thesis style is called for the body content

%----------------------------------------------------------------------------------------
%	TITLE PAGE
%----------------------------------------------------------------------------------------

\maketitlepage

%----------------------------------------------------------------------------------------
%	DEDICATION  (optional)
%----------------------------------------------------------------------------------------
%
%\dedicatory{For/Dedicated to/To my\ldots}
%\begin{dedicatory}

%\end{dedicatory}

%----------------------------------------------------------------------------------------
%	ABSTRACT PAGE
%----------------------------------------------------------------------------------------

\begin{abstract}
    Concept explanation has a great importance in languages with reduced lexicon that is not able to represent all the words of the Portuguese language.

    The users of LGP, when faced with words like 'Nanotechnology' have to resort to tools made for Portuguese because the ones available to them in LGP are not ideal.
    Those solutions, either lack translation features, which is the case of online dictionaries, or not practical, which is the case of sign language interpreters.

    To solve this problem, this project was created with the hypothesis of verifying if it's possible to utilize Text Mining, Information Scraping and Information Retrieval techniques to generate the explanation of a given word or expression that does not exist in the LGP lexicon.

    The planned solution consists in an Application Programming Interface (API) capable of generating an explanation of a given word or expression.
    This API will feed a web application responsible for receiving the user input and presenting the explanation in plain text and it's translation to LGP using an avatar.

    The solution is capable of displaying explanations in plain text, generated with the mentioned techniques, and exhibiting a clear indicator of their LGP readability for a deaf user.

\end{abstract}

\begin{abstractotherlanguage}
   A explicação de conceitos tem um papel importante em línguas com léxicos reduzidos, como é o caso das línguas gestuais.

    Em Portugal, a língua oficial utilizada pela comunidade surda, e as comunidades circundantes, tem o nome de Língua Gestual Portuguesa (LGP).
    Esta língua, como outras línguas gestuais, tem um léxico bastante reduzido o qual não é capaz de representar todas as palavras do Português.

    Os utilizadores da LGP, quando são confrontados com palavras como Nanotecnologia têm de recorrer a ferramentas criadas para Português, uma vez que as ferramentas disponíveis para LGP não uma solução ideal.
    Estas soluções, ou apresentam falhas na funcionalidades de tradução, que é o caso dos dicionários online, ou não são praticas, como é o caso dos intérpretes de língua gestual.

    Para dar resposta a este problema, este projecto foi criado com a hipótese de verificar se é possível utilizar técnicas de \textit{Text Mining}, \textit{Information Retrieval} e \textit{Information Scraping}, para gerar a explicação de uma palavra ou expressão que não faça parte do léxico da LGP.

    A solução proposta consiste numa \textit{Application Programming Interface} (API) que é capaz de gerar a explicação de uma dada palavra ou expressão.
    Esta API irá comunicar com uma aplicação web que é responsável por receber os \textit{inputs} do utilizador e apresentar as explicações geradas em texto, assim como a tradução para LGP utilizando um avatar.

    A solução final é capaz de apresentar, em texto, as explicações geradas utilizando as técnicas mencionadas assim como exibir de forma clara, ao utilizador surdo, o indicador de legibilidade de LGP associado a cada explicação.

\end{abstractotherlanguage}

%----------------------------------------------------------------------------------------
%	ACKNOWLEDGEMENTS (optional)
%----------------------------------------------------------------------------------------

\begin{acknowledgements}
    Firstly, to Instituto Superior de Engenharia do Porto for all the knowledge it allowed me to get and all the support it's teachers provided, with special focus on Dr. Nuno Escudeiro for being my supervisor in this project.

    To my best friend Tiago Soares, for all the support, advice and company, through all the happiness and sadness.Words will never be enough to repay your friendship.
    Also for following your dreams and taking the academic path I'm now here finishing. Always remember, "May we get what we want, may we get what we need, but never what we deserve".

    To my friend Miguel Carneiro, for all the knowledge, help, good conversations and for the motivation to always seek to improve myself.

    To my friends Carlos Figueiredo, Diogo Monteiro, Pedro Rodrigues and João Guedes, for all the moments we shared during the praxis and late night gatherings.

    Finally, and most importantly, to my parents Artur Dias and Maria Fernanda Dias, for all the sacrifices they made so I was able to have a safe and healthy home, even in dire times.
    This allowed me to have the time and space to pursue a higher education, which was an opportunity they never had.

    To all of you, my deepest and most sincere thank you,

    André Dias

\end{acknowledgements}

%----------------------------------------------------------------------------------------
%	LIST OF CONTENTS/FIGURES/TABLES PAGES
%----------------------------------------------------------------------------------------

\tableofcontents % Prints the main table of contents

\listoffigures % Prints the list of figures

\listoftables % Prints the list of tables

%\iflanguage{portuguese}{
%\renewcommand{\listalgorithmname}{Lista de Algor\'itmos}
%}
%\listofalgorithms % Prints the list of algorithms
%\addchaptertocentry{\listalgorithmname}


\renewcommand{\lstlistlistingname}{List of Source Code}
\iflanguage{portuguese}{
\renewcommand{\lstlistlistingname}{Lista de C\'odigo}
}
\lstlistoflistings % Prints the list of listings (programming language source code)
\addchaptertocentry{\lstlistlistingname}


%----------------------------------------------------------------------------------------
%	ABBREVIATIONS
%----------------------------------------------------------------------------------------
%\begin{abbreviations}{ll} % Include a list of abbreviations (a table of two columns)
%%\textbf{LAH} & \textbf{L}ist \textbf{A}bbreviations \textbf{H}ere\\
%%\textbf{WSF} & \textbf{W}hat (it) \textbf{S}tands \textbf{F}or\\
%\end{abbreviations}

%----------------------------------------------------------------------------------------
%	SYMBOLS
%----------------------------------------------------------------------------------------

%\begin{symbols}{lll} % Include a list of Symbols (a three column table)

%$a$ & distance & \si{\meter} \\
%$P$ & power & \si{\watt} (\si{\joule\per\second}) \\

%%Symbol & Name & Unit \\

%\addlinespace % Gap to separate the Roman symbols from the Greek

%$\omega$ & angular frequency & \si{\radian} \\

%\end{symbols}



%----------------------------------------------------------------------------------------
%	ACRONYMS
%----------------------------------------------------------------------------------------

\newcommand{\listacronymname}{List of Acronyms}
\iflanguage{portuguese}{
\renewcommand{\listacronymname}{Lista de Acr\'onimos}
}

%Use GLS
\glsresetall
\printglossary[title=\listacronymname,type=\acronymtype,style=long]

%----------------------------------------------------------------------------------------
%	DONE
%----------------------------------------------------------------------------------------

\mainmatter % Begin numeric (1,2,3...) page numbering
\pagestyle{thesis} % Return the page headers back to the "thesis" style
